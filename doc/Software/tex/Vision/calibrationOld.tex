\documentclass[11pt,a4paper,twocolumn]{article}
\usepackage[latin1]{inputenc}
\usepackage{german}
\usepackage{graphicx}
\usepackage{rotating}
\usepackage{geometry}

\title{Kalibrierungsanleitung tribots\_console}
\author{Sascha Lange}


\geometry{
  body={15.8cm, 24.3cm},
  left=2.8cm,
  top=2.5cm
}

\sloppy

\begin{document}
\maketitle

\section*{Schritt 1: Blende}
Die Blende der Kamera ganz aufdrehen, um die Belichtungszeit m�glichst kurz
halten zu k�nnen.

\section*{Schritt 2: Kamera einrichten}

coriander starten und folgende Einstellungen im
Reiter ``Features'' vornehmen:\\
\begin{tabular}{|l|l|}
\hline
Parameter     & Wert \\
\hline
brightness    & 128 \\
\hline
saturation    & 128 \\
\hline
hue           & 128 \\
\hline
auto exposure & 128 (Man) \\
\hline
white balance & auto \\
\hline
sharpness     & 0 \\
\hline
gamma         & 128 \\
\hline
shutter       & auto \\
\hline
gain          & auto \\
\hline
\end{tabular}\\
Dann im Reiter ``Camera'' im Feld ``Memory channels'' Setup1 ausw�hlen und
Save dr�cken. Die Kameraeinstellungen werden von der alten Software beim
Programmstart aus Memory Channel 1 geladen.

\section*{Schritt 3: Maske erstellen und Bildmittelpunkt}
Falls n�tig muss nun mit dem tool ImageMask
der neuen Software eine Maske zum ausmaskieren der Roboterhardware aus dem
Bild erstellt werden. Im Programm robot\_test
kann man sehen, ob die Erstellung einer neuen Maske notwendig ist: Treffen
die Scanlinien auf das Kameragestell oder die sonstige Teile des Roboters,
hat sich die Kamera vermutlich verschoben und die Erstellung einer neuen 
Maske ist notwendig.

Hierzu wird nun das Werkzeug ImageMask in der
neuen Software gestartet. Den Roboter sollte man so platzieren, dass sich
in seinem Sichtfeld ein deutlicher Farb�bergang befindet. Die einfachste 
L�sung ist es, den Roboter auf eine Teppichkante oder so weit wie m�glich
in ein Tor zu stelln. Dann dr�ckt man auf den Knopf capture. Der Roboter 
dreht sich nun f�r ein paar Sekunden und die Bilder werden aufgezeichnet. 
Die Bild-zu-Bild Abweichung des Farbwertes eines jeden Pixels wird w�hrend
dieses Vorgangs gemittelt. Die Idee ist, dass sich die Farbe der Pixel, die
zum Roboter geh�ren, w�hrend der Drehung kaum �ndern. Mit dem Regler links
kann man nun einen guten Schwellwert einstellen (die Streben der 
Kamerahalterung sollten nicht unterbrochen sein). Ggf. kann man den Vorgang
einfach wiederholen, um ein besseres Ergebnis zu erzielen. 

Nun speichert man die erzeugte Maske als ppm und tr�gt den Dateinamen in der
Konfigurationsdatei beim Feld {\it robot\_mask} ein (automatisch geht dies nur 
in der neuen Architektur).

\section*{Schritt 4: Distanzkalibrierung}

Beschreibung fehlt noch. Stefan W.?

\section*{Schritt 5: Farben einstellen}

Zuerst mal robot\_test starten.
Nun m�ssen nacheinander mindestens die Farben rot, weiss, gr�n richtig 
eingestellt werden (bei der neuen Architektur wird gr�n nicht beachtet). 
Schwarz f�r die Hindernisse ist auch ratsam. Die eingezeichneten gr�nen 
Kreise geben dar�ber Aufschlu�, wo mit den derzeitigen Einstellungen 
Linien gesehen werden.

Schritt f�r Schritt Anleitung:

\begin{enumerate}
\item Lookuptabelle ausschalten (damit die Regler aktiviert werden)
\item Load dr�cken (um die Reglerstellungen zu laden)
\item Farbe ausw�hlen. Beim ersten mal nach dem Laden erst auf eine andere
      Farbe und dann wieder aus Weiss schalten, da beim Laden die Regler nicht
      richtig initialisiert werden.
\item Die Regler komplett aufziehen (min nach links, max nach rechts)
\item Nun den Bereich so einschr�nken, dass nur noch die richtigen Pixel
      klassifiziert werden. Zuerst den Winkel (Farbwert) dann den Radius
      (S�ttigung) und zuletzt die Helligkeit einstellen, um Schwarz und Weiss
      zu trennen. Bei Farben l�sst man den S�ttigungsregler f�r max meist
      ganz rechts und schiebt nur den min-Regler. Bei Nichtfarben verf�hrt
      man genau andersherum.
\item Mit Punkt 3 fortfahren, bis alle Farben eingestellt sind.
\item Save dr�cken, um die Reglerstellungen f�r sp�tere Nachkalibrierungen
      abzuspeichern.
\item Lookuptabelle anschalten (erst hierduch wird die Datei lastyuv.lut mit
      den neuen Reglerstellungen erzeugt).
\item Programm verlassen.
\end{enumerate}

\section*{Schritt 6: ScanLine-Algorithmus ausw�hlen}

Es gibt mehrere ScanLine-Algorithmen, die je nach Umgebungsbedingungen 
besser oder schlechter funktionieren. Derzeit ist auf allen Robotern 
die ``new-irgendwas'' eingestellt und useDemoLineTransitionDetector ist auf
false. Diese Einstellung sollte in den Konfigurationsdateien nicht ohne
weiteres ge�ndert werden, da sonst auch eine �nderung der Schwellwerte f�r
die �berg�nge n�tig ist. Was der genaue Unterschied der einzelnen 
Algorithmen ist: Derzeit kein Ahnung!




\end{document}