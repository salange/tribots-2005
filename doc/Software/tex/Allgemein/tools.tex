
\section{Tools}
\label{sec:tools}

\subsection{RemoteClient}
\label{subsec:remoteclient}

\textit{RemoteClient} ist ein Programm, das auf einem Standrechner gestartet
zur Steuerung und Kommunikation mit dem Notebook auf dem Roboter
dient. Nach Aufbau einer WLAN-Verbindung durch Auswahl des Rechners,
auf dem \textit{robotcontrol} gestartet wurde, wird eine
UDP-Verbindung aufgebaut.

NN



\subsection{Tribotsview}
\label{subsec:tribotsview}

Das Programm \textit{Tribotsview} dient der Visualisierung
abgespeicherter Trajektorien und Debuginformation. Das Programm wird
mit zwei Argumenten gestartet: (a) dem Pr�fix der Logdateien sowie (b)
der Konfigurationsdatei, die zumindest die Spielfeldgeometrie
enthalten muss. Nach dem Start wird das Tribotsview-Fenster angezeigt,
das aus der Darstellung des Feldes, der textuellen Debugausgabe sowie
der Statusinformationsleiste besteht. Nun kann die aufgezeichnete
Trajektorie Zyklus f�r Zyklus abgefahren werden.

In der Felddarstellung werden verschiedene Informationen angezeigt,
die wahlweise ein- und ausgeschaltet werden k�nnen:
\begin{itemize}
\item Roboter und Robotergeschwindigkeit (translatorisch) in t�rkis
\item im Weltmodell verwalteter Ball mit Ballgeschwindigkeit in rot
\item im Weltmodell verwaltete Hindernisse in schwarz
\item durch die Bildverarbeitung erkannte Liniensegmente als hellgraue
  Kreise markiert
\item durch die Bildverarbeitung erkannten Ball als dunkelrot
  gebrochener Kreis
\item durch die Bildverarbeitung erkannte Hindenisse in grau
\item durch die Bildverarbeitung erkannte Tore als gelber und blauer
  Kreis markiert
\item Suchpositionen der Selbstlokalisierung in dunkelblau. Diese
  Funktion ist optional und kann in verschiedener Weise genutzt werden
  z.B. zur Markierung von Suchpositionen der Bildverarbeitung oder
  fr�heren Ballpositionen. Angezeigt werden jeweils die Informationen
  aus den *.ppos-Dateien
\item Spur des Roboters als t�rkise gebrochene Linie
\item Spur des Balls als rote gebrochene Linie
\item die in der *.log-Datei eingetragenen Linienelemente
\end{itemize}
Dar�ber hinaus k�nnen mit \textsl{Shift-Rechte Maustaste} Punkte und
Linien eingezeichnet werden, die erst durch Bet�tigen des Symbols
"`violette Tonne"' gel�scht werden.

Durch die Pr�diktion von Roboter- und Ballposition durch das
Weltmodell entsteht eine Diskrepanz zwischen dem Zeitpunkt
derDarstellung und dem Alter der Bildinformationen. Dadurch kann es zu
leichten Verschiebungen kommen, vor allem bei schnellen
Rotationsbewegungen. Durch Bet�tigen des Uhrsymbols werden Roboter-
und Ballposition um diese zeitliche Diskrepanz zur�ckgerechnet.

Auf der Statusleiste im rechten Teil des Tribotsfensters befindet sich
eine kreisrunde Anzeige, die drei Informationen darstellt: die
Ausrichtung des Roboters, die Rotationsgeschwindigkeit sowie die
Richtung des Balls relativ zum Roboter. 

\textit{Tribotsview} erm�glicht das Versetzen des Roboters mit der
Maus auf eine beliebige Position des Spielfeldes
(\textsl{Control-Linke Maustaste}). Diese Funktion dient dem Testen
alternativer Roboterpositionen. Eine solche Positions�nderung kann
durch Bet�tigen des B�roklammersymbols dauerhaft gespeichert werden,
so dass die ge�nderte Position bei vor- und zur�ckschalten nicht
verloren geht.

Die Steuerung von \textit{Tribotsview} ist auch per Tastatur
m�glich. Folgende Befehle stehen zur Verf�gung:
\begin{center}
\begin{tabular}{ll}
\hline
Taste & Funktion \\
\hline
+ & vergr��ern \\
- & verkleinern \\
a & gesamtes Feld darstellen \\
Backspace & zu letztem Bildausschnitt zur�ckkehren \\
Return & Bildausschnitt wiederherstellen (Redo) \\
$\rightarrow$,$\leftarrow$,$\uparrow$,$\downarrow$ & Bildauschschnitt
verschieben \\
\hline
Space & Trajektorie anhalten \\
p (previous) & letzten Zyklus anzeigen \\
n (next) & n�chsten Zyklus anzeigen \\
g (go) & Trajektorie vorw�rts abspielen \\
b (backward) & Trajektorie r�ckw�rts abspielen \\
f (fast forward) & Trajektorie vorspulen \\
r (rewind) & Trajektorie zur�ckspulen \\
\hline
t (trace) & Roboterspur an/abstellen \\
\hline
\end{tabular}
\end{center}

Mit Hilfe der Maus sind folgende Kommandos m�glich:
\begin{center}
\begin{tabularx}{\textwidth}{lX}
\hline
Taste & Funktion \\
\hline
Linke Maustaste & Bildausschnitt verschieben \\
Rechte Maustaste & Ausschnittsvergr��erung \\
Shift-Linke Maustaste & Koordinaten einer Position, Richtung und L�nge
einer Strecke anzeigen \\
Shift-Rechte Maustaste & Punkte oder Strecken dauerhaft markieren \\
Control-Linke Maustaste & Roboter versetzen \\
Control-Rechte Maustaste & SL-Fenster �ffnen \\
\hline
\end{tabularx}
\end{center}

Mit Hilfe des SL-Fensters kann der Slebstlokalisierungsalgorithmus
optimiert werden. F�r den ausgew�hlten Ausschnitt wird die
Fehlerlandschaft der Selbstlokalisierung angezeigt, d.h. die
Diskrepanz zwischen gesehenen Liniensegmenten und Modelllinien. Helle
Stellen markieren einen geringen Fehler, dunkle einen gro�en. Es wird
stets von einer optimalen Orientierung des Roboters ausgegangen. Der
t�rkise Kreis markiert die gegenw�rtige Roboterposition, der blaue die
optimale Position. Der magenta Kreis gibt die Position an, die durch
einige Optimierungsschritte aus der t�rkisen hervorgegangen ist. Durch
Klicken in das SL-Fenster wird der Roboter an die entsprechende Stelle
versetzt. \textsl{Revert all} berechnet die Fehlerlandschaft neu,
\textsl{Revert pos} �bernimmt eine neue Roboterposition aus dem
Hauptfenster ohne die Fehlerlandschaft neu zu berechnen.

